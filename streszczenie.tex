\addcontentsline{toc}{chapter}{Streszczenie}

Niniejsza praca skupiona jest na badaniach oddziaływań w układach w fazie ciekłej, poprzez interpretację wyników symulacji dynamiki molekularnej (MD). Badane układy stanowią niedawno badane eksperymentalnie potencjalne elektrolity do baterii nowej generacji. Stąd, zazwyczaj zawierały one jony. W~większości rozważanych przypadków zastosowano zarówno klasyczną MD jak i~dynamikę ab initio (AIMD).

Analizę rozpoczęto od właściwości strukturalnych w rozdziale~\ref{chapter:structural-md}. Jedną z grup badanych układów były roztwory soli w cieczach jonowych (IL): NaFSI w EMIM-FSI oraz Na/LiTFSI w EMIM-TFSI. Wpływ oddziaływań na wewnętrzną strukturę układu badano głównie za pomocą radialnych funkcji rozkładu (RDF) wybranych par atomów oraz analizy konformerów i~zmian ich liczebności przy zmieniającym się stężeniu elektrolitu. Na podstawie RDF wykryto preferencje koordynacyjne jonów metali oraz określono różnice w~strukturach koordynacyjnych dla sodu i litu, które porównano z wartościami eksperymentalnymi. Dynamikę układów badano za pomocą funkcji autokorelacji czasu oddziaływania, wykazując, że wymiana rozpuszczalnika w powłoce solwatacyjnej kationu staje się wolniejsza wraz ze wzrostem stężenia soli oraz że zmiany dla Na$^{+}$ są szybsze niż dla Li$^{+}$.

Drugi zestaw układów stanowiły elektrolity oparte na cieczach molekularnych: roztwory soli magnezu w dimetoksyetanie (DME) różniące się stosunkiem jonów Mg:Cl, Li/NaTFSI w węglanie etylenu (EC) i~jego fluorowanych pochodnych oraz roztwory NaFSI/TFSI w trzech rozpuszczalnikach eterowych: monoglimie, tetraglimie i poli(tlenku etylenu) (PEO). W~przypadku elektrolitów magnezowych wyznaczono struktury możliwych agregatów oraz ich energie. Dla układów o~różnym początkowym rozkładzie jonów klasyczne symulacje MD~zapewniły wgląd w~proces agregacji. Określono liczby koordynacyjne i~różnice między sferami koordynacyjnymi różnych jonów dla roztworów na bazie węglanów i eterów. W przypadku tych ostatnich zbadano preferencje dla różnych miejsc wiązania oraz ruchliwość kationów wzdłuż łańcuchów PEO.

Trzecią klasę układów stanowiły elektrolity z wodą: mieszaniny EMIM-TFSI/woda oraz stężone roztwory LiTFSI/woda. Dla EMIM-TFSI ze zmieniającym się stężeniem wody wyznaczono całkowitą liczbę wiązań wodorowych oraz ich rozkład pomiędzy różnymi donorami i akceptorami.

AIMD była szczególnie przydatna do wyznaczania widm w~podczerwieni (IR). Otrzymane wyniki opisano w rozdziale~\ref{chapter:ir-spectra}. Zwykle w porównaniu z~eksperymentem częstotliwości w widmach obliczonych na podstawie symulacji są systematycznie przesunięte. Jednak przesunięcia pasm występujące, gdy zmienia się skład układu (np. stężenie soli) są odtwarzane poprawnie. Przesunięcia wywołane przez oddziaływania zostały wyraźnie zaobserwowane dla oddziaływań anionów FSI$^{-}$ z kationami sodu oraz dla solwatacji Li$^{+}$/Na$^{+}$ w cyklicznych węglanach. Wpływ tworzenia wiązań wodorowych na widma IR jest również zadowalająco opisany, co można zaob-serwować dla układów EMIM-TFSI/H$_2$O i LiTFSI/H$_2$O.

W~celu uzyskania bardziej szczegółowego opisu drgań w~badanych układach zastosowa-no proste, ale skuteczne podejście. Polegało ono na obliczeniu transformat Fouriera (FT) poszczególnych parametrów geometrycznych, m.in. długości wiązań lub kątów. Takie FT pozwoliły przypisać drgania pasmom widocznym w widmie IR. Wykazano również, że FT pomagają ujawnić efekty oddziaływań, np. w~roztworach NaTFSI/LiTFSI w~EC zaobserwowano dwie odrębne grupy częstotliwości odpowiadające cząsteczkom rozpuszczalnika oddziałującym lub nieoddziałującym z kationami metali. Ponadto średnie FT dla wszystkich jonów lub cząsteczek wykazywały podobne wzorce przesunięć pozycji maksimów między układami o różnych składach, jak zaobserwowano w widmach IR.

Symulacje AIMD są kosztowne obliczeniowo, dlatego niektóre tańsze podejścia zostały opisa-ne w rozdziale~\ref{chapter:alternatives-to-aimd}. Dla dwóch układów badanych za pomocą AIMD wykonano dodatkowo symulacje metodą DFTB. Dla czystego EC i~jego pochodnych parametryzacja 3ob była w stanie odtworzyć przesunięcia pasma C=O w~widmie IR, jednak efekt był słabszy niż w~AIMD, a dla elektrolitu NaTFSI/EC nie udało się odtworzyć pojawienia się nowego pasma obserwowanego w~AIMD. Wariant GFN2-xTB umożliwił odtworzenie tego efektu dla elektrolitów litowych. Badanie przeprowadzone dla wody wykazało, że jakość uzyskanych wyników silnie zależy od zastosowanej parametryzacji i~jest najlepsza, gdy parametry są opracowywane dla konkretnego układu, chociaż również ogólne podejście GFN2-xTB dało zadowalające odwzorowanie widma. Jednak w przypadku układów ciecz jonowa-H$_2$O nawet podejście GFN2-xTB było niedokładne w opisie wiązań wodorowych. Zatem metoda DFTB jest obiecującą alternatywą dla AIMD, ale wciąż wymaga opracowania lepszych parametrów do tego celu.

Innym podejściem, tańszym obliczeniowo niż AIMD, były metody uczenia maszynowego. Dla czystego EC i~czystej wody sieci neuronowe zostały wytrenowane w~przewidywaniu energii potencjalnej, sił i momentu dipolowego dla danej geometrii układu. Możliwe było przeprowadzenie symulacji MD dla niewielkiej liczby cząsteczek, otrzymane widma były porównywalne z wynikami AIMD.

Przedstawione tu wyniki wykazały, że AIMD jest w~stanie nie tylko badać wpływ oddziaływań na strukturę elektrolitów, ale także odtwarzać ich przejawy w~widmach IR. Dalszy rozwój metodologii jest konieczny, aby wykorzystać potencjał podejść mniej wymagających obliczeniowo.
