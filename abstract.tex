\addcontentsline{toc}{chapter}{Abstract}

This work concentrates on studies of interactions in different systems, all in the liquid phase, by interpretation of Molecular Dynamics (MD) simulations. Investigated systems have recently been experimentally studied as potential electrolytes for novel batteries. Thus, in most cases there were some ions present in the liquid. In the majority of systems considered, both classical and ab initio (AIMD) MD approaches were used.

The analysis started with the structural properties in Chapter~\ref{chapter:structural-md}. One group of investigated systems were salt solutions in ionic liquids (ILs): NaFSI salt in EMIM-FSI IL and Na/LiTFSI salts in EMIM-TFSI IL. The effects of interactions on the internal structure of the system were studied mainly by radial distribution functions (RDFs) of selected atom pairs and the analysis of conformers and changes in their abundance for changing concentrations of the electrolyte. On the basis of RDFs, coordination preferences of metal ions were detected, and differences in coordination patterns for sodium and lithium were determined and compared to experimental values. Dynamics of the systems was studied via residence time autocorrelation functions showing that solvent exchange in the solvation shell of the cation becomes slower with increasing salt concentration and that the changes for Na$^{+}$ are faster than for Li$^{+}$.

The other set of systems were electrolytes based on molecular liquids: solutions of magnesium salts in dimethoxyethane (DME) with different Mg:Cl ions ratio, Li/NaTFSI in ethylene carbonate (EC) and its fluorinated derivatives, and NaFSI/TFSI solutions in three ethereal solvents: monoglyme, tetraglyme, and poly(ethylene oxide) (PEO). In the case of Mg~electrolytes structures of possible aggregates and their energies were determined. For systems with different initial distribution of ions, classical MD~simulations provided insight into the aggregation process. The coordination numbers and differences between the coordination shells of different ions were studied for carbonate- and ether-based solutions. For the latter, preferences for different binding sites and the mobility of cations along PEO chains were examined.

The third class of systems were electrolytes with water: EMIM-TFSI/water mixtures and concentrated LiTFSI/water solutions. For EMIM-TFSI IL with changing concentration of water, the total number of hydrogen bonds was determined, as well as their distribution between different donors and acceptors.

AIMD was particularly useful for the determination of infrared (IR) spectra. The results obtained are described in Chapter~\ref{chapter:ir-spectra}. Usually, when compared to experiment, frequencies in the spectra calculated from simulations are systematically shifted. However, band shifts occuring when the composition of the system (e.g. salt concentration) changes are reproduced correctly. The interaction-induced shifts were clearly demonstrated for interactions of FSI$^{-}$ anions with sodium cations and for Li$^{+}$/Na$^{+}$ solvation in cyclic carbonates. The effect of HB formation on the IR spectra is also satisfactorily described, as can be seen for EMIM-TFSI/H$_2$O and LiTFSI/H$_2$O systems.

For a more detailed description of systems, a~simple but successful approach was used. It was based on the calculation of Fourier transforms (FTs) of particular geometrical parameters, e.g. bond lengths or angles. Such FTs allowed to assign vibrations to bands visible in IR spectrum. It was also demonstrated that FTs help reveal effects of interactions; e.g.~in NaTFSI/LiTFSI solutions in EC, two separated groups of frequencies were observed corresponding to solvent molecules interacting or non-interacting with metal cations. In addition, the averages of the FTs over all ions or molecules exhibited similar patterns of shifts in the positions of the maxima between systems with different compositions, as observed in the IR spectra.

AIMD simulations are computationally expensive, therefore, some cheaper approaches were described in Chapter~\ref{chapter:alternatives-to-aimd}. For two of the systems studied by AIMD additionally DFTB MD simulations were performed. For neat EC and its derivatives, the 3ob~parametrization was able to reproduce shifts of the C=O band in the IR spectrum; however, the effect was weaker than in AIMD, and for the NaTFSI/EC electrolyte, it failed to reproduce appearance of a~new band observed in AIMD. The GFN2-xTB approach was able to reproduce this effect for lithium electrolytes. A study carried out for bulk water showed that the quality of the results obtained strongly depends on the parameterization used and is the best when the parameters are developed for a~particular system, although also the general-purpose GFN2-xTB approach yielded satisfactory reproduction of the spectrum. However, for IL-H$_2$O case even the GFN2-xTB approach was inaccurate in the description of HB. Thus, the DFTB method is a~promising alternative to AIMD, but still needs the development of better parameters for this purpose.

Another approach computationally cheaper than AIMD were Machine Learning methods. For neat EC and for neat water, neural networks were trained to predict the potential energy, forces, and dipole moment for a~given system geometry. It was possible to perform MD simulations for a~small number of molecules, resulting spectra were comparable to AIMD results.

The results reported here demonstrated that AIMD is able not only to study the effects of interactions on the structure of electrolytes, but also to reproduce their manifestations in the IR~spectra. Further development of methodology is necessary to exploit the potential of computationally less demanding approaches.
