\section{Aim of this work}

This work concentrates on studying the relationship between interactions in solutions (mostly electrolytes) and the structure of the system or its IR spectrum. In particular, its aim is to examine whether AIMD is able to reproduce effects of such interactions on vibrational spectra observed experimentally. Furthermore, an analysis of correlations between the calculated spectrum of the whole system and the local environment of molecules is done. In contrast to most of the works mentioned in Section~\ref{sec:vibrational_spectra_from_AIMD}, here changes in IR spectrum with changing composition of a~particular system are studied.

As AIMD is rather computationally expensive, the quality of results obtained by simplified methods was also studied. These methods were DFTB and a~novel approach with utilization of neural networks.

To summarize, the main aims could be briefly divided into the following points:
\begin{itemize}
    \item studying the structure (preferred complexes, coordination numbers, hydrogen bonds) and the dynamics (stability of aggregates, residence times) of electrolytes and liquids,
    \item obtaining theoretical IR spectra from AIMD simulations, comparing them with experimental data, and assignment of spectral features to effects of interactions in the solution,
    \item finding and analysing correlations between position of bands in the IR spectrum and the chemical environment of molecules or ions,
    \item testing the applicability of approaches computationally cheaper than AIMD, such as Density Functional based Tight-Binding (DFTB) or Machine Learning (ML) of interatomic potentials.
\end{itemize}

Theoretical methods used in this thesis will be presented in Chapter~\ref{chapter:theoretical-methods}. Results will be described and discussed in Chapters~\ref{chapter:structural-md}-\ref{chapter:alternatives-to-aimd}.