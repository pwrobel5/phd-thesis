This Thesis focused on liquid phase systems (ionic liquids and electrolytes) of relevance to the research on new metal-ion batteries. Special attention was paid to the changes in the properties of the solution related to the change in the concentration of the dissolved substance (salt, water) and to the effects of interactions, in particular those observable in the IR spectra. The methodology used in this work was based on molecular dynamics (MD) simulations, both classical and ab initio. Structural analysis of the results involved mostly calculations of Radial Distribution Functions (RDFs) and residence time autocorrelation functions for specific ion pairs. For chosen systems, theoretical IR spectra were obtained from MD trajectories as Fourier transforms (FTs) of the dipole moment autocorrelation functions. FTs of selected geometrical parameters (interatomic distances, angles) were used in the analysis of vibrational modes.

Structural analysis of the data revealed conformational preferences, structure of aggregates, coordination numbers for cation-anion or cation-solvent complexes, and their dependence on the concentration of the solution. Experimentally observed differences between complexation patterns for lithium and sodium cations were successfully reproduced, as well as experimental predictions about dominating forms of complexes in Mg-based electrolytes for different ratios of Mg:Cl ions were confirmed.

Residence time autocorrelation functions, determined from FF~MD allowed to observe the dynamics of exchange of atoms in the coordination shell of cations. It was shown that the exchange process for Na$^{+}$ cations is faster than for Li$^{+}$ ions. When the concentration of dissolved salt increases, the dynamics of the solvation shell becomes slower.

AIMD simulations for ionic liquid-water mixtures and water-in-salt electrolytes were able to produce patterns of hydrogen bonding in agreement with experiment, which was confirmed by analysis of calculated IR spectra compared to experimental data.

A~major part of this work describes the modeling of IR spectra. It was concluded that AIMD is capable of reproduction of measured changes caused by changing interactions with changing composition of the system --- red/blue shifts of bands or appearance of new bands in the spectrum. Analysis of FTs of individual geometrical parameters supported the explanation of observed effects and assignment of particular vibrational modes to related bands in the IR spectrum. This approach allowed to correlate the frequency of the vibration with chemical environment of molecules, e.g.~interactions with ions or hydrogen bond formation. The advantage of this method is that it is possible to apply a~posteriori to a~recorded trajectory in opposition to Wannier function analysis, which requires determination of Wannier centers during the MD simulation.

Applicability of two approaches: DFTB and Machine Learning (ML), computationally chea-per than AIMD, was studied for reproduction of IR spectra. Although some results were close to those obtained from AIMD simulations, overall performance was not satisfactory. For DFTB there is a~need of improved parametrizations; here, it was shown that specifically developed parameter sets (e.g.~for water) could have much better performance. From general purpose parametrizations GFN2-xTB was the most valuable one, but still needs improvements. ML methods have problems with stability of simulated systems, which may be the effect of insufficient training sets and needs further investigation. Nevertheless, both approaches seem as promising alternatives to AIMD.

To conclude, FF~MD (with some limitations) and AIMD are powerful tools for the analysis of liquid phase systems. The latter can be applied to study the vibrational spectra and their sensitivity to interactions, yielding results comparable to those obtained by experiment. Less time-consuming approaches are not yet accurate enough to replace AIMD, but after their development in the future, they may become a~reasonable alternative.

\section{Acknowledgements}

The simulations performed for the purpose of this work were done with use of the PlGrid infrastructure: Prometheus and Ares clusters in the Academic Computer Centre Cyfronet AGH within computational grants plgaenael2, plgaenael3, plgaenael4, plgaenael5, plgaenael6, plgaenael7, plgaegpu3 and plgaegpu4.

This work was supported by the National Science Centre grant no. UMO-2016/21/B/St4/ 02110, by Jagiellonian University POB DigiWorld minigrant no. U1U/P06/DO/14.07 and by the Jagiellonian Interdisciplinary PhD Programme in the ZintegrUJ project.